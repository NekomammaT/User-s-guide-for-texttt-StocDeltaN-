% !TEX TS-program = xelatex%

\documentclass[aps, prd
, preprint
%, twocolumn
, nofootinbib 
, notitlepage
%, superscriptaddress
, longbibliography
]{revtex4-1}
\usepackage{graphicx}
\usepackage[caption=false]{subfig}
\usepackage{mathrsfs}
\usepackage{amsmath,amssymb}
\usepackage{bm}
\usepackage{braket}
\usepackage{listings}
\usepackage{cases}
\usepackage{comment}
\usepackage{soul}
\usepackage{cancel}
\usepackage{cases}
\usepackage[utf8]{inputenc}
\usepackage{url}
\usepackage{longtable}
\usepackage[normalem]{ulem}
\usepackage[colorlinks=true
,urlcolor=blue
,anchorcolor=blue
,citecolor=blue
,filecolor=blue
,linkcolor=blue
,menucolor=blue
%,pagecolor=blue
,linktocpage=true
,pdfproducer=medialab
,pdfa=true
]{hyperref}

%\usepackage{mathpazo}
%\usepackage[no-math]{fontspec}
%\setmainfont{Palatino}
%\setsansfont{Optima}

\newcommand{\dif}[2]{\frac{\mathrm{d} #1}{\mathrm{d} #2}}
\newcommand{\pdif}[2]{\frac{\partial #1}{\partial #2}}
\newcommand{\var}[2]{\frac{\delta #1}{\delta #2}}
\newcommand{\dd}{\mathrm{d}}
\newcommand{\DD}{\mathscr{D}}
\newcommand{\ee}{\mathrm{e}}
\newcommand{\diag}{\mathrm{diag}}
\newcommand{\sgn}{\mathrm{sgn}}
\newcommand{\Mpl}{M_\text{Pl}}
\newcommand{\ns}{n_{{}_\mathrm{S}}}
\newcommand{\cs}{c_{{}_\mathrm{S}}}
\newcommand{\IR}{\text{IR}}
\newcommand{\UV}{\text{UV}}
\renewcommand{\Re}{\mathrm{Re}}
\renewcommand{\Im}{\mathrm{Im}}
\newcommand{\dk}{\frac{\dd^3k}{(2\pi)^3}}
\newcommand{\bbalpha}{{\alpha\!\!\!\alpha}}
\newcommand{\dps}{\displaystyle}
\newcommand{\SIA}{S_\text{IA}}
\newcommand{\eff}{\text{eff}}
\newcommand{\kdx}{\mathbf{k}\cdot\mathbf{x}}

\newcommand{\calD}{\mathcal{D}}
\newcommand{\scrD}{\mathscr{D}}
\newcommand{\calg}{\mathcal{g}}
\newcommand{\calH}{\mathcal{H}}
\newcommand{\scrH}{\mathscr{H}}
\newcommand{\uI}{\text{I}}
\newcommand{\calJ}{\mathcal{J}}
\newcommand{\scrJ}{\mathscr{J}}
\newcommand{\calL}{\mathcal{L}}
\newcommand{\scrL}{\mathscr{L}}
\newcommand{\calN}{\mathcal{N}}
\newcommand{\calO}{\mathcal{O}}
\newcommand{\scrO}{\mathscr{O}}
\newcommand{\calP}{\mathcal{P}}
\newcommand{\calR}{\mathcal{R}}
\newcommand{\uR}{\text{R}}

\newcommand{\bae}[1]{\begin{align} #1 \end{align}}
\newcommand{\bce}[1]{\begin{cases} #1 \end{cases}}
\newcommand{\bfe}[4]{
\begin{figure} 
	\centering
	\includegraphics[#1]{#2}
	\caption{#3}
	\label{#4}
\end{figure}}
\newcommand{\bpme}[1]{\begin{pmatrix} #1 \end{pmatrix}}

\newcommand{\Red}[1]{\textcolor{red}{\sffamily #1}}
\newcommand{\Mag}[1]{\textcolor{magenta}{\sffamily #1}}
\newcommand{\Blue}[1]{\textcolor{blue}{\sffamily #1}}
\newcommand{\mathblue}[1]{\textcolor{blue}{#1}}



\begin{document}
\title{User's guide for StocDeltaN \\[-10pt] {- \small\textit{field-space type} -}}
\date{\today}

\author{Yuichiro Tada}
\email{tada.yuichiro@e.mbox.nagoya-u.ac.jp}
\affiliation{Department of Physics, Nagoya University, Nagoya 464-8602, Japan}
%\affiliation{Institut d'Astrophysique de Paris, GReCO, UMR 7095 du CNRS et Sorbonne Universit\'e, 
%98bis boulevard Arago, 75014 Paris, France}


%\begin{abstract}
%\end{abstract}

\maketitle
%\tableofcontents


\section{Code Overview}

StocDeltaN is a powerful C++ package to analyze inflationary dynamics and calculate the power spectrum of curvature perturbations with use of the techniques of 
the stochastic-$\delta N$ approach~\cite{Fujita:2013cna,Vennin:2015hra}. We provide two types of StocDeltaN: one is the full phase-space formulation time without the slow-roll approximation,
while the other is the field-space formulation omitting the degrees of freedom (d.o.f.) of inflaton momenta under the slow-roll approximation.
Though the latter, which is labeled by ``\_conf" \Blue{(meaning of ``configuration". to be renamed?)}, sometimes wrong in models where e.g. the slow-roll conditions are violated for an instant,
it is much more economic computationally thanks to the halved d.o.f. and often enough for leading order calculations.
In this user's guide, we focus on this field-space type though the usage is naturally extended to the phase-space type.

StocDeltaN consists of two parts as ``source" part containing all required numerical solvers which we provide and ``main" part where users specify the inflationary model
and the usage of several options like a plotting option with use of Python. Users can write the main code using various sample codes as references.

\newpage

``Source" part in itself is divided into three parts as \texttt{JacobiPDE\_conf.cpp}, \texttt{SRK32\_conf.cpp}, and \texttt{StocDeltaN\_conf.cpp}.
\texttt{JacobiPDE\_conf.cpp} and \texttt{SRK32\_conf.cpp} implement the solver classes for partial differential equations (PDE) and stochastic differential equations (SDE) respectively.
Overriding these classes, user can use the numerical solver for general problems beyond the inflationary system if want.
\texttt{StocdeltaN\_conf.cpp} overrides their classes and defines the specified class as the stochastic-$\delta N$ solver.
The main code specifies the inflationary model, overriding its functions defining e.g. potential, field-space metric, and so on,
and then user can use its member functions to analyze the stochastic inflation.


\section{Tutorial}

\subsection{Prerequisties}

\begin{itemize}
\item {\sffamily\bfseries C++ compiler} : We have checked the operation in GNU, Clang, and Intel compiler on Mac system. GNU compiler will be used by default.
On Mac system, generally Clang compiler is preinstalled and automatically called instead of GNU compiler, so users need not to install other compilers by themselves. 
By modifying Makefile, one can change the used compiler if wants.

\item {\sffamily\bfseries Make} : For a clear compilation, StocDeltaN employs Make which is a compilation manager. On Mac system, it is easy to introduce GNU Make
by installing Command Line Tools of Xcode.

\item {\sffamily\bfseries OpenMP (optional)} : 

\item{\sffamily\bfseries Python (optional)} : 
\end{itemize}




\section{Code Structures}




%\acknowledgments





%\appendix







\bibliography{main}
\end{document}